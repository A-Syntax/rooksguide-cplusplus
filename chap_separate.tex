% This work by Jeremy A. Hansen is licensed under a Creative Commons 
% Attribution-NonCommercial-ShareAlike 3.0 Unported License, 
% as described at http://creativecommons.org/licenses/by-nc-sa/3.0/legalcode

Separate compilation is the process of breaking a C++ program into separate files to improve organization. 
Parts of the program can be spread out over a number of different files that are later compiled individually, then \Keyword{linked} using a linker to produce the final, working program. 
When changes are made, only those files with changes need to be recompiled, the result of which can then be relinked with the previously-compiled files. 
This process is nearly invisible in most development environments, which recompile and relink these files automatically.
When the development environment takes care of these details, the user is left with the sole task of making changes where they are needed. 

One of the most basic applications of separate compilation is used when writing abstract data types. 
Recall from Chapter \ref{chap_classes} that there are declaration and definition sections in a \Code{class}. 
The declaration contains class functions and variables, both public and private, while the definition section is where the function definitions and most actual code can be found. 
The process of separate compilation requires the two sections to be split into separate files, each of which is written and maintained separately and later used together to create a working program. 

Declarations will be put into the \Keyword{interface} file or the \Keyword{header file} which typically has a \Code{.h} suffix.
In most code written by novice programmers, there will be only one \Code{class} declaration in each header file. 
To use the \Code{class} in your code elsewhere, you should use \Code{\#include} followed by the file name in double quotes. 
Here is an example of the contents of an interface file called \Code{student.h}:

\noindent\begin{minipage}{\linewidth}\begin{lstlisting}
#include <string>
using namespace std;

class student
{
public:
  student();
  int getAge();
  void setAge(int update);
  int getID();
  void setID(int update);
  string getName();
  void setName(string update);
private:
  int age;
  int ID;
  string name;
};
\end{lstlisting}\end{minipage}

To use the \Code{student} class in some other source code file, that file should include the following line:

\noindent\begin{minipage}{\linewidth}\begin{lstlisting}
#include "student.h" 
\end{lstlisting}\end{minipage}

The quotes around \Code{student.h} tell the compiler to find the header file in the same directory as the current file.

The \Keyword{implementation} file will include all the function definitions for the \Code{student} class. 
The implementation file can be called anything the programmer wants, but typically ends with a \Code{.cpp} suffix. 
For example, the implementation file for \Code{student} will probably be \Code{student.cpp}.

To ensure that a new implementation file is compiled into your program, you do not need to \Code{\#include} anything. 
However, the development environment will automatically compile and link the implementation file if it has been added to your project.
The only files that you should \Code{\#include} are header files. 

To avoid linker errors, your files should have safeguards to ensure that classes and functions are not declared more than once within the same program. 
These safeguards are simple, and should be included in each header file. 
For example, we place the following two lines at the top of the file \Code{student.h}:

\noindent\begin{minipage}{\linewidth}\begin{lstlisting}
#ifndef STUDENT_H // STUDENT_H could be anything 
#define STUDENT_H // as long as it is unique to this file
\end{lstlisting}\end{minipage}

The following line should go at the end of the same file: \nopagebreak[4]

\noindent\begin{minipage}{\linewidth}\begin{lstlisting}
#endif //STUDENT_H - a reminder about the #ifndef above 
\end{lstlisting}\end{minipage}

\noindent The above three lines do the following:

\begin{enumerate}
\item Test if \Code{STUDENT\_H} has been previously \Code{\#define}d, usually because this header file has been \Code{\#include}d elsewhere.
\item If it has not been \Code{\#define}d, \Code{\#define} it now and proceed with compiling the code between the \Code{\#ifndef} and \Code{\#endif}.
\item Close the \Code{\#ifndef} block. If \Code{STUDENT\_H} was previously defined, skip to the line after this one.
\end{enumerate}

\noindent Here is an example of what these lines look like alongside some actual code:

\noindent\begin{minipage}{\linewidth}\begin{lstlisting}
#ifndef STUDENT_H
#define STUDENT_H

class student
{
//class declaration because this is an interface file
};

#endif//STUDENT_H
\end{lstlisting}\end{minipage}

This combination of preprocessor directives will ensure that the \Code{student} class is only defined once.


\LevelD{Review Questions}

\begin{enumerate}
	\item What is a header file?

	\item What file extension do we typically use for a C++ header file?

	\item What file extension do we typically use for a C++ implementation file?

  \item In which file would would you typically store an abstract data type's (ADT's) declaration?

	\item How do you incorporate a header file named \Code{something.h} into a file named \Code{main.cpp}?

	\item Do you incorporate an implementation file into your project the same way?

	\item How do you prevent redeclaration of ADTs and functions in header files?
\end{enumerate}

\LevelD{Review Answers}

\begin{enumerate}
	\item A header file stores the interface of an ADT

	\item A header file ends in \Code{.h}

	\item An implementation file ends in \Code{.cpp}

	\item In the interface file

	\item Add \Code{\#include "something.h"} alongside the other \Code{\#include} statements in \Code{main.cpp}.

	\item No, the implementation file will automatically be compiled and linked by your development environment as long as the implementation file is in your project.

	\item You prevent redeclaration by adding lines similar to the following to the top of your header file:

\noindent\begin{minipage}{\linewidth}\begin{lstlisting}
#ifndef SOMETHING_H
#define SOMETHING_H
\end{lstlisting}\end{minipage}

\noindent Then add the following to the end of the header file:

\noindent\begin{minipage}{\linewidth}\begin{lstlisting}
#endif // SOMETHING_H
\end{lstlisting}\end{minipage}

\end{enumerate}

\LevelD{Further Reading}

\begin{itemize}
\item \url{http://elm.eeng.dcu.ie/~ee402/ee402notes/html/ch03s14.html}
\item \url{http://web-ext.u-aizu.ac.jp/~fayolle/teaching/2012/C++/pdf/1-separate_compilation.pdf}
\end{itemize}	
