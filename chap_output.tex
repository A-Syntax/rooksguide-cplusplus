% This work by Jeremy A. Hansen is licensed under a Creative Commons 
% Attribution-NonCommercial-ShareAlike 3.0 Unported License, 
% as described at http://creativecommons.org/licenses/by-nc-sa/3.0/legalcode

Output in C++ is done with the object \Code{cout} (``\textbf{c}onsole \textbf{out}put''). 
The object \Code{cout} prints useful information to the screen for the user. 
For example, if we wanted to prompt the user with 

\noindent \Code{Type in your name:}

\noindent we would use \Code{cout}. 
\Code{cout} is extremely important when you are starting to learn C++ as it gives you the ability to display the current state of any variable and provide user feedback at any point in your program. 
Let's make a program that outputs something to the screen:

\noindent\begin{minipage}{\linewidth}\begin{lstlisting}
#include <iostream>
using namespace std;
int main()
{
  cout << "Go Cadets!";
  return 0;
}
\end{lstlisting}\end{minipage}

%Don't mind the grayed out code, that's necessary, but we'll get to it later. 
%Your development environment may provide much of that code anyways; if not, type it in, but don't worry about that material just yet. 

The symbol \Code{<<} is called the \emph{redirect operator} and is needed between \Code{cout} and what you want to display to the screen. 
In this case, we are displaying a string literal \Code{"Go Cadets!"}. 
As you know, every statement in C++ ends with a semicolon, and this one is no exception.

What if we want to print more, though?

\noindent\begin{minipage}{\linewidth}\begin{lstlisting}
#include <iostream>
using namespace std;
int main()
{
  cout << "Go Cadets!";
  cout << "You can do it!";
  return 0;
}
\end{lstlisting}\end{minipage}

Try to compile and run that. 
It works, but it's not really the desired output. 
You should get:

\noindent \Code{Go Cadets!You can do it!}

How do we get those on a different line? 
One of the ways we can do it is to use the object \Code{endl}. 
\Code{endl} means ``\textbf{end} \textbf{l}ine'', and is used when you want to end one line and start over on the next---it's like hitting enter on your keyboard. 
You will also need another redirect operator between the string literal and the \Code{endl}. 
Putting all of this together looks like this:

\noindent\begin{minipage}{\linewidth}\begin{lstlisting}
#include <iostream>
using namespace std;
int main()
{
  cout << "Go Cadets!" << endl;
  cout << "You can do it!";
  return 0;
}
\end{lstlisting}\end{minipage}

\noindent This prints:

\noindent \Code{Go Cadets!}

\noindent \Code{You can do it!}

That works a bit more as intended. 
Alternatively, we can combine the two lines that use \Code{cout} into a single one like this:

\noindent\begin{minipage}{\linewidth}\begin{lstlisting}
cout << "Go Cadets!" << endl << "You can do it!";
\end{lstlisting}\end{minipage}

Another way we can accomplish this, without needing another redirect operator, is with the special character \Code{'\textbackslash n'}.
\Code{'\textbackslash n'} is a newline character, it prints a new line just like the \Code{endl} object. 

\noindent\begin{minipage}{\linewidth}\begin{lstlisting}
#include <iostream>
using namespace std;
int main()
{
  cout << "Go Cadets!\nYou can do it!";
  return 0;
}
\end{lstlisting}\end{minipage}

\noindent This prints:

\noindent \Code{Go Cadets!}

\noindent \Code{You can do it!}

Another thing we can use with the console output object is the \Code{'\textbackslash t'} character. 
Printing this character is the same as pressing the tab key on your keyboard, and is used for indentation and formatting. 
Let's look at an example that uses the newline character, the tab character, and some text:

\noindent\begin{minipage}{\linewidth}\begin{lstlisting}
#include <iostream>
using namespace std;
int main()
{
  cout << "\tGo Cadets!\nYou can do it!";
  return 0;
}
\end{lstlisting}\end{minipage}

\noindent This code prints: 

\Code{Go Cadets!}

\noindent \Code{You can do it!}

We don't always have to output words the screen using \Code{cout}. 
We can also print variables of type \Code{int}, \Code{double}, and \Code{float} and can control the number of digits that appear after the decimal point. 
For example, if we had a variable that contained the value 3.14159265 we might only care about the first two numbers after the decimal point and just want to output 3.14 to the screen. 
We do that with the \Code{precision()} member function. 
This function call will result in subsequent \Code{float} or \Code{double} variables being printed with the specified number of decimal places. 
In the following code, the number of digits is set to 2:

\noindent\begin{minipage}{\linewidth}\begin{lstlisting}
#include <iostream>
using namespace std;
int main()
{
  double num = 3.14159265;
  cout.precision(2);
  cout << num << endl;
}
\end{lstlisting}\end{minipage}

\noindent This code prints:

\noindent \Code{3.14}

To display data in a similar way as a spreadsheet, we can create a field of characters and set the number of characters in each field using the \Code{width()} and \Code{fill()} member functions. 
Notice the use of the \Code{left} flag in the following code, which positions the output on the left side of the field; the default is for the output to be on the right side:

\noindent\begin{minipage}{\linewidth}\begin{lstlisting}
#include <iostream>
using namespace std;
int main()
{
  cout << "Norwich" << endl;
  cout.width(15);
  cout << "University" << endl;
  cout.fill('*');
  cout.width(20);
  cout << left << "Corps of Cadets" << endl;
}
\end{lstlisting}\end{minipage}

\noindent The above code prints:

\noindent \Code{Norwich}

\noindent \Code{~~~~~University}

\noindent \Code{Corps of Cadets*****}

\LevelD{Review Questions}
\begin{enumerate}
	\item Which of the following is a correct way to output \Code{Hello World} to the screen?
	  \begin{enumerate}
	  \item \Code{output: "Hello World";}
	  \item \Code{cout >> "Hello World";}
	  \item \Code{cout << "Hello World";}
	  \item \Code{console.output << "Hello World";}
	  \end{enumerate}
  \item Which of the following is a correct way to output \Code{Hello!} to the screen on one line and \Code{Goodbye!} to the screen on the next line?
		\begin{enumerate}
		\item \Code{cout >> "Hello!" >> "Goodbye!";}
		\item \Code{output: "Hello!\textbackslash nGoodbye!";}
		\item \Code{cout << "Hello!" << \textbackslash n << "Goodbye!";}
		\item \Code{cout << "Hello!" << '\textbackslash n' << "Goodbye!";}
		\end{enumerate}
  \item Aside from the answer in the previous question, write two alternative ways of printing \Code{Hello!} and \Code{Goodbye!} to the screen on two different lines.
	\item Write several lines of code using the \Code{width()} and \Code{fill()} functions in a \Code{main()} that prints \Code{Programming!} to the screen with 4 \Code{'x'} characters printed after it.
	\item Write code to output the values 124, 12.376, \Code{z}, 1000000, and \Code{strings!} as distinct values, separated by spaces.
	\item What is the output of the following program?

\noindent\begin{minipage}{\linewidth}\begin{lstlisting}
#include <iostream>
#include <string>
using namespace std;
int main()
{
  string shirt = "maroon", pants = "blue";

  cout << shirt << " " << pants << endl;
  return 0;
}
\end{lstlisting}\end{minipage}

\end{enumerate}

\LevelD{Review Answers}
\begin{enumerate}
	\item c.
	\item d.
	\item \Code{cout << "Hello!" << endl << "Goodbye!";} or
	
	\Code{cout << "Hello!\textbackslash nGoodbye!";} 
	
	(other similar answers are possible)
	\item
\noindent\begin{minipage}{\linewidth}\begin{lstlisting}
cout.fill('x');
cout.width(16);
cout << left << "Programming!";
\end{lstlisting}\end{minipage}
	\item \Code{cout << 124 << " " << 12.376 << " z " << 1000000 <<  " strings!";}
	\item \Code{maroon blue}
\end{enumerate}

\LevelD{Further Reading}

\begin{itemize}
\item \url{http://www.java-samples.com/showtutorial.php?tutorialid=245}
\item \url{http://www.cplusplus.com/doc/tutorial/basic_io}
\item \url{http://www.cplusplus.com/reference/ostream/ostream/}
\item \url{http://www.cplusplus.com/doc/tutorial/functions/}
\end{itemize}	

