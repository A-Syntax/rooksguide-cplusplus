% This work by Jeremy A. Hansen is licensed under a Creative Commons 
% Attribution-NonCommercial-ShareAlike 3.0 Unported License, 
% as described at http://creativecommons.org/licenses/by-nc-sa/3.0/legalcode

\LevelD{Literals}

A literal is a value outside of a variable such as $5$, $9$, $103$, and $-21$. 
Each of those is an \Code{int}, but a literal can be of any data type. 
The point is, these are values that the C++ compiler already recognizes, and can't be changed. 
In other words, you can't convince the compiler to give the literal 3 the value of 4, because 3 is constant. 
Table \ref{table-literal-examples} contains a few examples.

\begin{table}
	\centering
		\begin{tabular}{| c | c |}
		\hline
			\textbf{Literal value} & \textbf{Data Type} \\ \hline
			\Code{123.45f} & \Code{float} \\ \hline
			\Code{13.8903} & \Code{double} \\ \hline
			\Code{-389283220.342423} & \Code{double} \\ \hline
			\Code{49e-8} & \Code{double} \\ \hline
			\Code{12}    & \Code{int} \\ \hline
			\Code{12u}    & \Code{unsigned int} \\ \hline
			\Code{'x'} & \Code{char} \\ \hline
		  \Code{"text"} & \Code{string} \\ \hline
			\Code{true} & \Code{bool} \\ \hline
			\Code{false} & \Code{bool} \\ \hline
			

		\end{tabular}
  \caption{Examples of a few literals} \label{table-literal-examples}
\end{table}

\LevelD{Declared Constants}

We call a variable whose value we cannot change a constant. 
After you declare a constant, you are unable to change it, no matter what. 
The difference between declaring a normal variable and a constant is that we simply place the keyword \Code{const} before the data type in the declaration. 
This indicates whatever variable and type that follows the \Code{const} will be a constant and cannot be changed. 
Since it is a constant, we will also need to initialize the value at the time we declare the variable. 
Here is an example (we cover the \Code{cout} object shortly in Chapter \ref{chap_output}): \nopagebreak[4]

\noindent\begin{minipage}{\linewidth}\begin{lstlisting}
const float pi = 3.14;
float radius = 5, area;

area = radius * radius * pi;
cout << area; // outputs 78.5
\end{lstlisting}\end{minipage}

\LevelD{Review Questions}

\begin{enumerate}
	\item Describe the difference between literals and declared constants. 
	When would a declared constant be more useful than a literal constant?
	\item What is the difference between a normal variable and a constant?
%	\item Write code that uses a \Code{const} to add 3\% interest to a bank account every year. Print out the balance for the next ten years.
	\item Build a program in C++ that does the following:
	
	\begin{enumerate}
		\item Declare a \Code{double} variable named \Code{Feet}. Initialize it to your height.
		\item Declare a \Code{double} constant named \Code{MetersPerFoot}, with the value of $0.3048$.
		\item Declare a \Code{double} variable named \Code{Meters}. Set it to \Code{Feet} multiplied by \Code{MetersPerFoot}.
%		\item	Display \Code{Meters} to the screen.
	\end{enumerate}

	\item Create a program that displays the diameter and area of a circle for any given radius. Use a \Code{const float} to represent $\pi$. 

\end{enumerate}

\LevelD{Review Answers}

\begin{enumerate}
	\item A literal is a value not stored in a variable; a constant is an unchanging value stored in a variable.
	\item Normal variables can be changed or overwritten; constants cannot be changed or overwritten.
	\item 
\noindent\begin{minipage}{\linewidth}\begin{lstlisting}
double Feet 5.5;
const double MetersPerFoot = .3048;
double Meters = Feet * MetersPerFoot;
\end{lstlisting}\end{minipage}

	\item
\noindent\begin{minipage}{\linewidth}\begin{lstlisting}
float radius = 5;
const float pi = 3.14159
double diameter, area;
diameter = radius * 2;
area = pi * (radius * radius)
\end{lstlisting}\end{minipage}


\end{enumerate}


%\LevelD{Further Reading}
%\begin{itemize}
%\item \url{~}
%\item \url{~}
%\end{itemize}
