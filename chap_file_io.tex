% This work by Jeremy A. Hansen is licensed under a Creative Commons 
% Attribution-NonCommercial-ShareAlike 3.0 Unported License, 
% as described at http://creativecommons.org/licenses/by-nc-sa/3.0/legalcode

\Keyword{File I/O} refers to the input and output (I/O) from and to files. 
So far we have been using \Code{cin} to get input from the keyboard and \Code{cout} to output to the screen. 
Just like output can be sent to the screen, output can be sent to a file. 
Input can be taken either from a keyboard or from a file. 
Input and output is handled in the program through objects called \Keyword{streams}. 
This chapter will discuss how to take input from a file and send output to the same file or a different one. 

File I/O is useful because files provide a way to store data permanently. 
With keyboard input and screen output, the data is temporary and goes away once the program is finished.
When it comes to files, the data is there for us and we do not have to waste our time typing it over and over again. 

\LevelD{I/O Streams}

If data is flowing into your program it is called an \Keyword{input stream}. 
If data is flowing out of the program it is called an \Keyword{output stream}. 
We have actually been using both types of streams already! 
\Code{cin}, which handles a flow of data from the keyboard, is an input stream and \Code{cout}, which produces a flow of data to the screen, is an output stream. 
If an input stream object is connected to a file, then the program can get its input from that file. 
Similarly, an output stream object can send data to the screen or to a file. 
A file can be opened for both reading and writing, in which case it can be accessed by both input and output streams.

\LevelD{File I/O}

When the program opens a file for input, the program is reading from the file. 
When the program opens a file for output, the program is writing to the file. 
C++ provides us with the \Code{ifstream}, \Code{ofstream}, and \Code{fstream} classes for reading from and writing to files. 
All of these classes are available through the \Code{fstream} library, which means we must \Code{\#include} it in our code in order to use them:

\noindent\begin{minipage}{\linewidth}\begin{lstlisting}
#include <fstream>
\end{lstlisting}\end{minipage}

The \Code{ofstream} type (read that as ``\textbf{o}utput \textbf{f}ile \textbf{stream}'') is used to write data to files. 
The \Code{ifstream} type (``\textbf{i}nput \textbf{f}ile \textbf{stream}'') is used to read data from files. 
Objects of type \Code{fstream} (``\textbf{f}ile \textbf{stream}'') can combine the behavior of \Code{ifstream} and \Code{ofstream} and allow us to both read from and write to files.

The \Code{cin} and \Code{cout} objects are already declared for you. 
However, in order to use \Code{ifstream}, \Code{ofstream} and \Code{fstream} objects, you must declare one like you would any other variable. 
Declaring these objects looks like this:

\noindent\begin{minipage}{\linewidth}\begin{lstlisting}
//Declares a variable of type ifstream named input
ifstream inFile; 
//Declares a variable of type ofstream named output
ofstream outFile; 
\end{lstlisting}\end{minipage}

The variable \Code{inFile} will deal with getting input from a file, while the variable \Code{outFile} will deal with outputting data to a file. 

Every file on a computer has its own name and a location (or \Keyword{path}).
An example of a text file name is \Code{TextFile.txt} and its location in a Windows operating system might be \Code{c:\textbackslash storage\textbackslash TextFile.txt}. 
In a UNIX-based operating system, the same file might be in \Code{/home/user1/TextFile.txt}. 
Regardless of the operating system, we need to know the file's path in order to tell the program where to find the file. 

\LevelD{Opening and closing a File}

Before we can even start reading from and writing to a file we must open it. 
In order to open a file you must first make an object of type \Code{ifstream}, \Code{ofstream}, or \Code{fstream} just like we did earlier. 
We open a file using a member function named \Code{open}. 
The \Code{ofstream} object will create a file for you if the file you're opening for output does not exist.
Otherwise, if the file already exists, the \Code{open} function will erase existing data in the file by default. 
The following example demonstrates how to open files for both input and output:

\noindent\begin{minipage}{\linewidth}\begin{lstlisting}
#include <iostream> //For cin and cout 
#include <fstream> // For ifstream and ofstream
using namespace std;
int main ()
{
  //Declares a variable of type ifstream called inFile
  ifstream inFile; 
  //Declares a variable of type ofstream called outFile
  ofstream outFile; 
	
	//Opens text file for input
  inFile.open("TextFile.txt"); 
  //Creates text file for output
  outFile.open("OutputTextFile.txt"); 
	
  return 0;
}
\end{lstlisting}\end{minipage}

%TODO: Add an example of appending data to an ofstream

Once you are done with the file, it is good practice to close it. 
Closing the file disconnects it from the program and prevents the program from continuing to read from or write to the file. 
If the program ends normally or crashes, the files will be automatically closed. 
Closing files is even simpler than opening them. 
All you need to do is use the \Code{close} function with empty parentheses. 
For example, to close both \Code{inFile} and \Code{outFile}:

\noindent\begin{minipage}{\linewidth}\begin{lstlisting}		
inFile.close();
outFile.close();
\end{lstlisting}\end{minipage}
	
\LevelD{Reading from a File}

We use the \Code{ifstream} class to read data from a file. 
Instead of having a user input data from the keyboard, we now input the data from a file. 
As you recall from earlier in the book, we used \Code{cin} with \Code{>>}, the \Keyword{extraction operator}. 
This is the operator we use when we would like get input from the keyboard and it is also used with \Code{ifstream} objects. 
Once we have declared our variable of type \Code{ifstream} and opened a file, we can use it to input data. 
Using this is very similar to \Code{cin} except we replace \Code{cin} with the name of our variable.
For example: \nopagebreak[4]

\noindent\begin{minipage}{\linewidth}\begin{lstlisting}		
#include <iostream> 
#include <fstream>
using namespace std;
int main()
{
  int number = 5;
  ifstream inFile;

  inFile.open("TextFile.txt");

  inFile >> number; 
  // The value 5 in number is overwritten
	// by the integer stored in the file

  return 0;
}
\end{lstlisting}\end{minipage}		

This will read in one integer from the file and store it into the variable \Code{number}. 
You can input all different types of data including characters, \Code{double}s, and \Code{float}s. 
Overall, \Code{ifstream} objects are very similar to \Code{cin}---you just have to declare one and remember to use the variable name instead of \Code{cin}.

\LevelD{Writing data to a File}

We use the \Code{ofstream} class to output data to files. 
\Code{cout} outputs data to our screen whereas \Code{ofstream} stores data in files. 
Just like \Code{cout}, \Code{ofstream} objects use \Code{<<}, the \Keyword{insertion operator}. 
Using this is very similar to \Code{cout} except we replace the \Code{cout} with the name of our variable. 
For example: \nopagebreak[4]

\noindent\begin{minipage}{\linewidth}\begin{lstlisting}		
#include <iostream> 
#include <fstream>
using namespace std;
int main ()
{
  char Letter = 'A';
  ofstream outFile;

  outFile.open("OutputTextFile.txt");

  outFile << Letter; // Puts the letter 'A' into the file

  return 0;
}
\end{lstlisting}\end{minipage}

This example would write the letter \Code{'A'} to the text file we created named \Code{OutputTextFile.txt}. 
You can also create numeric variables and output them to the file just like:
			
\noindent\begin{minipage}{\linewidth}\begin{lstlisting}
int num = 10;
outFile << num << endl;
\end{lstlisting}\end{minipage}		

This example would output the number 10 and create a new line in the text file we created. 

\LevelD{Introduction to Classes and Objects}
	
We will go into more detail about classes and objects in Chapter \ref{chap_classes} but it is necessary to go over it briefly in this section. 
Both \Code{cin} and \Code{cout} are objects. 
An \Keyword{object} is a variable that has functions built in and may have multiple pieces of data associated with it. 
\Code{ifstream} and \Code{ofstream} are object types that define which operations may be performed on and which data are stored in the objects. 
For example, the function \Code{open()} (along with \Code{close()} and many others) is considered a \Keyword{member function} of \Code{ifstream} and \Code{ofstream}, which means it is a function that is associated with object of those two types. 
Getting a little more into detail, these object types are defined as part of a \Keyword{class}. 
A \Code{class} is a blueprint for complex data types. 
We already know data types such as integers, \Code{double}s, and \Code{char}s, but using classes, you will be able to design your own data type.

When calling the functions \Code{open} or \Code{close}, you will notice we use a period between the object name and the function. 
We call this the \Keyword{dot operator} and it is used to reference member functions and member variables of a class. 

\LevelD{Other functions}

The \Code{<fstream>} library comes with many functions to help test to see if things are working. 
One example is the \Code{fail()} function. 
We use this function to determine whether the file was opened successfully or not. 
We usually use \Code{if} statements with the function so that if the file does not open correctly we can warn the user. 
For example:

\noindent\begin{minipage}{\linewidth}\begin{lstlisting}
inFile.open("TextFile.txt");
if (inFile.fail())
{ 
  cout << "Failed to open!";
}
\end{lstlisting}\end{minipage}

This will warn the user if the file did not open correctly. 
If the file \emph{did} open correctly, the program would continue without printing the error message.

The next function is the \Code{eof()} (\textbf{e}nd \textbf{o}f \textbf{f}ile) function. 
This function tests to see if the stream has reached the end of the file. 
This function is very useful in order to know when to stop reading from the file. 
For example:

\noindent\begin{minipage}{\linewidth}\begin{lstlisting}
int number;
inFile.open("TextFile.txt");
while (!inFile.eof())
{
  inFile >> number
}
\end{lstlisting}\end{minipage}

This example shows how the \Code{eof()} function can be used in a \Code{while} loop. 
The \Code{while} loop will read integers from the file until the program reaches the end of the file. 
This is useful for gathering all the data from one file. 

The \Code{get()} and \Code{put()} functions are used to read and write single characters, respectively. 
The function \Code{get()} allows the program to read in a single character into a variable of type \Code{char}. 
When we use the \Code{>>} operator, spaces, tabs and newlines---the whitespace characters---around data are skipped automatically. 
However with \Code{get()}, nothing is done automatically, so the whitespace characters can be extracted, too. 
The member function \Code{get()} takes one argument in parentheses that must be a \Code{char} variable. 
For example:

\noindent\begin{minipage}{\linewidth}\begin{lstlisting}
char Character;
ifstream inFile;

cin.get(Character);
//or
inFile.get(Character);
\end{lstlisting}\end{minipage}

This will read in the next character typed on the keyboard or from the file. 
Even if the next character is a space, tab, or newline, the program will store that character in the variable. 

The \Code{put()} function is used to output one character. 
This function takes one argument of type \Code{char} in the parentheses. 
For example:

\noindent\begin{minipage}{\linewidth}\begin{lstlisting}
char Character = '\n'; // newline character
ofstream outFile;

cout.put(Character);
// or
outFile.put(Character);
\end{lstlisting}\end{minipage}

\LevelD{Review Questions}

\begin{enumerate}
\item What do we call the type of object used to control data flowing into your program?
\item What do we call the type of object used to control data flowing out of your program?
\item What header file must you \Code{\#include} in order to use \Code{ifstream} and \Code{ofstream}?
\item What are \Code{ifstream} and \Code{ofstream} used for?
\item How do you declare an \Code{ifstream} object named \Code{input} and an \Code{ofstream} object named \Code{output}?
\item How would you open a file named \Code{TextFile.txt} with an \Code{ifstream} object called \Code{input}?
\item How would you close a file named \Code{TextFile.txt} with an \Code{ofstream} object called \Code{output}?
\item What kind of function is the \Code{eof()} function and what does it do?
\item What are the benefits of using files for input and output?
\item What is the difference between \Code{cin~>>~c;} and \Code{cin.get(c);} if \Code{c} is of type \Code{char}?
\item Write a program that outputs the contents of some file to the screen.
\item Write a program that reads in a text file and prints to the screen the number of times the character \Code{'e'} shows up.
\end{enumerate}

\LevelD{Review Answers}

\begin{enumerate}
\item An input stream
\item An output stream
\item You need to \Code{\#include <fstream>}
\item \Code{ifstream} is used to read data from a file. \Code{ofstream} is used to write data to a file.
\item \Code{ifstream input;}

			\Code{ofstream output;}
\item \Code{input.open("TextFile.txt");}
\item \Code{output.close();}
\item The \Code{eof()} function is a member function. It returns \Code{true} if the program has reached the end of the file.
\item File input and output are useful because files provide a way to store data permanently. With keyboard input and screen output, the data is temporary and disappears once the program is finished. The data stored in files on the other hand remains the same until another program changes it. Also, an input file can be used by many programs at the same time without having to store multiple copies or re-enter the data over and over again.
\item The first \Code{cin} statement the next non-whitespace character into \Code{c}, but the call to \Code{cin.get()} stores the next character in \Code{c} whether it is whitespace or not.
\end{enumerate}

\LevelD{Further Reading}

\begin{itemize}
\item \url{http://www.cprogramming.com/tutorial/lesson10.html}
\item \url{http://www.cplusplus.com/doc/tutorial/files/}
\item \url{http://www.tutorialspoint.com/cplusplus/cpp_files_streams.htm}
\end{itemize}	

