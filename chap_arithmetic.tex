%Arithmetic

One of the most important things provided by C++ is the ability to do math. 
Everything a computer sees is a number. \
To a computer, its ability to do math and manipulate numbers is as essential to it as breathing is to us. 
(My apologies to anything not living that may be reading this).

The operators (\Code{+}, \Code{-}, \Code{*}, \Code{/}) in C++ are slightly different from what you may be used to from your second-grade math class. 
Addition is still a plus sign ( \Code{+} ) and subtraction is still a minus sign ( \Code{-} ). 
On the other hand, multiplication becomes an asterisk ( \Code{*} ) and division becomes a forward slash ( \Code{/} ). 
Think of it as \textit{over} as in ``5 over 9'' is the same as the fraction $5/9$.

To do math in C++, you will either want a variable to store the answer, or output the answer to the user. 

The following code directly outputs the answer to the user:

\begin{lstlisting}
cout << 9 + 2;
\end{lstlisting}

This code shows how to use a variable to store the answer:

\begin{lstlisting}
int sum = 9 + 2;
\end{lstlisting}

Note that when you use a variable to store an answer, the variable must come first in the equation (before the equal sign) and must be the only thing on the left side of the equation. 
There are some other things to note.
When you use more complicated equations, you can use parentheses to help. 
C++ uses a familiar order of operations (Parentheses, Exponents, Multiply, Divide, Add, and Subtract, or PEMDAS), but without the exponent operation (this topic is covered in Chapter \ref{Advanced Arithmetic}). 
However, unlike in normal arithmetic, parentheses do not imply multiplication. 
\Code{(4)(3)} does not mean the same as \Code{4~*~3}. 
\Code{(4)(3)} results in a syntax error and does not compile. 
The compiler returns an error message like this: \Code{'error: '4' cannot be used as a function.'}

In C++, there are several methods of shortening and simplifying the code you're creating. 
The first is the increment operator ( \Code{++} ), which is found in the name of the language, C++.
This operator increases the value of the variable it's applied to by 1. 
Conversely, the decrement ( \Code{--} ) operator decreases the value by 1.

Keep in mind that order does matter with the increment and decrement operators. 
They can be used as either prefixes or suffixes, but where you put the operator results in slightly different behavior. 
Starting with similarities, \Code{C++} and \Code{++C} both increase value of \Code{C} by one. 
The difference lies in when another variable is being set to that incremented value, such as \Code{B = C++}. 
\Code{B} will be set to \Code{C} before \Code{C} is incremented. 
\Code{B = ++C} will cause \Code{B} to be set to \Code{C+1}, in a similar way to \Code{B = 1 + C}.

\begin{lstlisting}
int A;
A = 4;
A++;
//A contains 5
\end{lstlisting}
\begin{lstlisting}
int A;
A = 9;
A--;
//A contains 8
\end{lstlisting}
\begin{lstlisting}
int A, B;
B = 7;
A = B++;
//A contains 7, B contains 8
\end{lstlisting}
\begin{lstlisting}
int A, B;
B = 7;
A = ++B;
//A contains 8, B contains 8
\end{lstlisting}
\begin{lstlisting}
int A, B;
B = 3;
A = B--;
//A contains 3, B contains 2
\end{lstlisting}
\begin{lstlisting}
int A, B;
B = 3;
A = --B;
//A contains 2, B contains 2
\end{lstlisting}

Compound assignment operators can decrease the amount you type and can make your code more readable.
These are the operators \Code{+=}, \Code{-=}, \Code{*=}, and \Code{/=}. 
What makes these operators special is that they use the value you want to change in the operation. 
For example, \Code{x += 2} is equivalent to \Code{x = x + 2}.

Keep in mind the order that was used, as this becomes important with subtraction and division. 
The variable being changed is equivalent to the two leftmost variables in the longhand expression.
Let's say we have \texttt{X} and \texttt{Y}, and want to set \texttt{X} equal to the value of \texttt{Y} divided by the value of \texttt{X}. 
This is impossible with this method, as \texttt{X /= Y} is equivalent to \texttt{X = X / Y}, and \texttt{Y /= X} is equivalent to \texttt{Y = Y / X}.

\begin{table}[tb]
	\centering
		\begin{tabular}{| c | c |}
		\hline
			\textbf{Expression} & \textbf{Equivalent to} \\ \hline
			\Code{A *= 3;} & \Code{A = A * 3;} \\ \hline
			\Code{B -= 5;} & \Code{B = B - 5;} \\ \hline
			\Code{C /= 10;} &	\Code{C = C / 10;} \\ \hline
		\end{tabular}
\end{table}

Here is some sample code using the concepts we presented in this chapter:

\begin{lstlisting}
#include <iostream>

using namespace std;

int main()
{
	int a = 5, b = 10, c = 15, d = 20;

	cout << "a + b = " << a + b << endl;
	cout << "d - c = " << d - c << endl;
	cout << "a * b = " << a * b << endl;
	cout << "d / a = " << d / a << endl;
}
\end{lstlisting}

The output of this code is:

\begin{lstlisting}
a + b = 15
d - c = 5
a * b = 50
d / a = 4
\end{lstlisting}


\LevelD{Review Questions}
\begin{enumerate}
\item Write a statement declaring two integer variables \Code{a} and \Code{b} and initialize them to 6 and 3, respectively. 
\item Without changing the last line, fix the following code so there will be an output of 12.
\begin{lstlisting}
	int a = 4, b= 2;
	a = a + 2 * b;
	cout << a;
\end{lstlisting}

\item What is the output of the following code?
\begin{lstlisting}
  int a=2, b=5, c=6;
  a++;
  b = b*a;
  c = (c-a) + 3;
  cout << a << endl;
	cout << b << endl;
	cout << c << endl;
\end{lstlisting}	
	
\item What is the output of the following code?
\begin{lstlisting}
  int a, b, c;
  a = 2;
  b = 8;
  c = 1;
  c = b - b;
  c = a + a;
  c = b * 8;
  b = b + b;
  c = a + c;
  b = a + b;
  a = a * c;
  b = a - c;
  c = b + a;
  cout << a << endl;
  cout << b << endl;
  cout << c << endl;
\end{lstlisting}
\end{enumerate}

\LevelD{Homework Exercises}
\begin{enumerate}
\item What is the output of the following code?
\begin{lstlisting}
  int a=4, b=2, c, d;
  a = b + 3;
  b++;
  c = (b + 4) * 2;
  c = c + 2;
  d = a + b - 3;
  a++;
  a = a + 2 - b;
  b = b * 2;
  cout << "a=" << a << endl;  
  cout << "b=" << b << endl;  
  cout << "c=" << c << endl;  
  cout << "d=" << d << endl; 
\end{lstlisting}

\item What is the output of the following code?
\begin{lstlisting}
  int m=3, n=2, x, y;
  x = m + 5;
  m--;
  y = (m+4) / 3;
  n = n + 2;
  m = m + n / 2;
  m++;
  x = x * 2 - 3;
  y = y * 2;
  n = n + y * 3
  cout << "m=" << m << endl; 
  cout << "n=" << n << endl;  
  cout << "x=" << x << endl;   
  cout << "y=" << y << endl; 
\end{lstlisting}
\end{enumerate}

\LevelD{Review Answers}
\begin{enumerate}
\item \Code{int a = 6, b = 3;}

\item
\begin{lstlisting}
  int a = 4, b = 2;
	a = (a + 2) * b;
	cout << a;
\end{lstlisting}

\item 
\begin{lstlisting}
3
15
5
\end{lstlisting}

\item
\begin{lstlisting}
132
66
198
\end{lstlisting}

\end{enumerate}

\LevelD{Homework Answers}
\begin{enumerate}
\item
\begin{lstlisting}
a=5
b=6
c=16
d=5
\end{lstlisting}

\item
\begin{lstlisting}
m=5
n=16
x=13
y=4
\end{lstlisting}
\end{enumerate}

\LevelD{Further Reading}
\begin{itemize}
\item \url{http://www.cplusplus.com/doc/tutorial/operators/}
\item \url{http://www.sparknotes.com/cs/c-plus-plus-fundamentals/basiccommands/section1.rhtml}
\end{itemize}