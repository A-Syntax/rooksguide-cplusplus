%Assignments

Assignments are a way for a user or a programmer to assign a value to a variable. 
The way we assign a value to a variable in C++ is different from how we might do it in math. 
In mathematics we are allowed to say that \Code{x = 3} or \Code{3 = x}, but in C++ the only acceptable way to assign the value of \Code{3} to \Code{x} is to type \Code{x = 3}. 

The \Code{=} in the expression \Code{x = 3} is known as an \Keyword{assignment operator}. 
This allows the program to set a variable's value depending on its type. 
Here are some examples of setting a value to different types of variables:

\noindent\begin{minipage}{\linewidth}\begin{lstlisting}
int x = 4;
\end{lstlisting}\end{minipage}

\noindent\begin{minipage}{\linewidth}\begin{lstlisting}
char alpha = 'A';
\end{lstlisting}\end{minipage}

\noindent\begin{minipage}{\linewidth}\begin{lstlisting}
string word = "Alpha";
\end{lstlisting}\end{minipage}

\noindent\begin{minipage}{\linewidth}\begin{lstlisting}
float y = 3.14;
\end{lstlisting}\end{minipage}

We are able to declare variables and assign a value to those variables immediately by using the assignment operator. 
When we assign literal values to variables of type \Code{char}, the value must be surrounded by single quotes (for example, \Code{'A'}). 
When we assign values to variables of type \Code{string}, the literal value must be surrounded by double quotes (for example, \Code{"Alpha"}). 
We do not have to initialize the values of the variables, however. 
We can set them later in the code like this:

\noindent\begin{minipage}{\linewidth}\begin{lstlisting}
int myVal;
//some code
myVal = 0; 
\end{lstlisting}\end{minipage}

In all of the lines of code in this section where a variable is set using the assignment operator, the ``thing that is being given a value'' is known as an \Keyword{lvalue}, and the expression on the right that is being stored in the variable is known as the \Keyword{rvalue}. 
Literals such as \Code{'A'} or \Code{3} can never be an lvalue. 
Aside from literals, the rvalue can consist of other variables, like this:

\noindent\begin{minipage}{\linewidth}\begin{lstlisting}
myVal = myVal2;
\end{lstlisting}\end{minipage}

Even though \Code{myVal2} is a variable, we are only using the \textit{value stored in the variable}, not the variable itself. 
For example, if \Code{myVal2} had a value of \Code{6}, \Code{myVal} would then be assigned to the value \Code{6} with the above code. 

We can also store the results of an arithmetic expression in a variable like this: 

\noindent\begin{minipage}{\linewidth}\begin{lstlisting}
myVal = 5 + 6; //assigns myVal a value of 11
\end{lstlisting}\end{minipage}

But we can't write

\noindent\begin{minipage}{\linewidth}\begin{lstlisting}
5 + 6 = myVal; // ERROR!
\end{lstlisting}\end{minipage}

\noindent since \Code{5 + 6} doesn't refer to a place where we can store a value. 
We can also combine arithmetic expressions with variables as an rvalue like this:

\noindent\begin{minipage}{\linewidth}\begin{lstlisting}
myVal2 = 6;
myVal = 4 + myVal2;
\end{lstlisting}\end{minipage}

In this case, the variable \Code{myVal} would be assigned a value of \Code{10} because the variable \Code{myVal2} was initialized to a value of \Code{6}, and \Code{4 + 6} is \Code{10}.
The value of \Code{myVal2} remains unchanged. 
Make sure that the variable \Code{myVal}, the variable \Code{myVal2}, and the literal \Code{4} are of the same type. 
For example, the following code will result in an error:

\noindent\begin{minipage}{\linewidth}\begin{lstlisting}
int myValue = 4;
int yourVal;
string myString = "word";
		
yourVal = myValue + myString; 
// Adding string to an int is
// probably not what you meant!
\end{lstlisting}\end{minipage}

When we try to combine different variable types, the compiler will get very mad at us. 
Some exceptions to this rule are if we try to combine \Code{float}s, \Code{int}s, and \Code{double}s.
These types have the ability to be combined (to a certain extent) because they are all numeric values.
Both \Code{double}s and \Code{float}s can hold values with a decimal point such as $-3.14$, $0.003$, or $5.167289$ whereas an \Code{int} can only hold round values such as $2$, $-18$, or $100$. 
Refer to Chapter \ref{chap_datatypes} for more information on converting between data types.

\LevelD{Review Questions}

\begin{enumerate}
	\item Which of the following is an incorrect way to assign a value to a variable \Code{x} of type \Code{int}?
	\begin{enumerate}
	  \item \Code{7 = x;}
		\item \Code{int x = 7;}
		\item \Code{int x(7);}
		\item \Code{x = 7;}
	\end{enumerate}

	\item Which of the following is an incorrect way to assign a value to a variable of type \Code{string}?
	\begin{enumerate}
		\item \Code{string myString = "word";}
		\item \Code{string myString = 'word';}
		\item \Code{myString = "word";}
	\end{enumerate}

 \item Is the following code incorrect? If so, why? If it is correct, why?

\noindent\begin{minipage}{\linewidth}\begin{lstlisting}
int x = 6, y;
char myChar = 'x';
y = myChar + x;
\end{lstlisting}\end{minipage}

  \item Write a program that declares two \Code{int} variables and two \Code{double} variables. 
  Add and subtract five from each of your declared integer variables. 
  Then add and subtract 7.32 your \Code{double} variables by $7.32$. 
  Then output each of your results to the screen. 
\end{enumerate}

\LevelD{Review Answers}

\begin{enumerate}
	\item \textbf{a.} When we store a value in a variable, the variable goes on the left of the assignment operator, and the value being stored in that variable goes to the right of the assignment operator.

  \item \textbf{b.} String literals must be surrounded by double quotes, not single quotes; single quotes are used for single characters like \Code{'b'}.

  \item The code is incorrect. This will probably not produce the expected result it tries to add an \Code{int} and a \Code{char} and store that value in a variable of type \Code{int}.

\end{enumerate}

\begin{comment}
\LevelD{Further Reading}

\begin{itemize}
\item ~
\item ~
\item ~
\end{itemize}	
\end{comment}