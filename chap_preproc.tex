Preprocessor directives are lines of code that are executed before the compilation of the code begins. 
These directives are like getting everyone in a room before starting a project or doing warmups before running a race. 
One of the most frequently-used preprocessor directives is \Code{\#include}.

When we want to include in our code a system library or some other file, we use the keyword \Code{\#include} followed by the library name or the file name. 
The way we distinguish between including libraries and including files is with angle brackets and quotes, respectively. 
For example, when we want to use objects like \Code{cout} or \Code{cin}, we need to include the \Code{iostream} library like so: 

\begin{lstlisting}
#include <iostream> 
\end{lstlisting}

If we want to include a file, such as a file named \Code{myFile.h}, we can write: 

\begin{lstlisting}
#include "myFile.h"
\end{lstlisting}

However, when we include files, they must be in the same directory as the file where the \Code{\#include} appears. 
We discuss the Standard Template Library in Chapter \ref{chap_stl}, and include a short sample of other libraries below:

\begin{table}[tb]
	\centering
		\begin{tabular}{| l | p{0.8in} | p{2in} |}
		\hline
			\textbf{Library} & \textbf{Provides} & \textbf{Some common uses} \\ \hline
			
			\Code{<iostream>} & Input/output stream objects & \Code{cout}, \Code{cin}: see Chapters~\ref{chap_input}~and~\ref{chap_output} \\ \hline
			\Code{<cstdlib>} & The C standard library & \Code{rand()}, \Code{abs()}, \Code{NULL} \\ \hline
			\Code{<math>} & Mathematical functions & \Code{pow()}, \Code{sqrt()}, \Code{cos()}, \Code{tan()}, \Code{sin()}: see Chapter~\ref{chap_advancedarith} \\ \hline
			\Code{<iomanip>} & Input/output manipulation & \Code{get\_money()}, \Code{get\_time()}, \Code{put\_time()} \\ \hline
			\Code{<ctime>} & Time-related functions & \Code{clock()}, \Code{time()}, \Code{ctime()} \\ \hline
			\Code{<string>} & The \Code{string} class & See Chapter~\ref{chap_string} \\ \hline
			\Code{<fstream>} & File input and output streams & See Chapter~\ref{chap_file_io} \\ \hline
				
		\end{tabular}
\end{table}

%TODO: #if, #ifdef, #ifndef, #define, etc.


\LevelD{Review Questions}

\LevelD{Homework Questions}

\LevelD{Review Answers}

\LevelD{Homework Answers}

\LevelD{Further Reading}

\begin{itemize}
\item \url{~}
\item \url{~}
\item \url{~}
\end{itemize}	



