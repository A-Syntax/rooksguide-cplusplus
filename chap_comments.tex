%Comments

As a C++ programmer, comments will make your life easier. 
They are meant to serve as notes, not just for you, but for anyone that may attempt to read your code. 
To this end, comments are a quick explanation of the code. 
There are two kinds of comments, \Keyword{single-line comments} and \Keyword{multi-line comments}.

Single-line comments typically come after a line of code. 
For a single-line comment, simply type a double slash \Code{//} at the end of the line, and write whatever notes you want after, preferably to explain what that line of code does.
Alternatively, the comment can start on a line of its own.
Here are some examples:

\begin{lstlisting}
	int count; // This variable was declared 
	           // to count something

	count = count + 1; //Increments count by 1

  //Variable declared, and initialized to pi
	float length=3.14159; 
\end{lstlisting}

Multi-line comments, sometimes called \Keyword{block comments}, are used when you have a lot to say. 
They begin with a slash star (\Code{/*}) and are ended by a star slash (\Code{*/}). 
Here is an example:

\begin{lstlisting}
/* 
 *This is a
 *multi-line 
 *comment
 */
 
/* This is also a comment */ 
\end{lstlisting}

Block comments do not need a star at the beginning of every line, but many programmers write it anyways, because it makes it easier to see and understand that ``this is still a comment, don't write code here.'' 
Some development environments will automatically color-code certain pieces of code, so comments might be gray, for example, and the \Code{*} at the beginning of each line might be unnecessary in that case. 
However, someone else may use a different development environment that does not use colors, so the stars can still improve readability.

Keep in mind when commenting, the point is to be clear and concise. 
Try to explain what's happening as accurately as possible, but try to keep it short. 
As you learn C++, use comments to explain what you're doing and why. 
You have to assume that the person reading your code needs an explanation for each non-trivial line. 