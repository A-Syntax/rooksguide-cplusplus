% This work by Jeremy A. Hansen is licensed under a Creative Commons 
% Attribution-NonCommercial-ShareAlike 3.0 Unported License, 
% as described at http://creativecommons.org/licenses/by-nc-sa/3.0/legalcode

%Comments

As a C++ programmer, comments will make your life easier. 
They are meant to serve as notes, not just for you, but for anyone that may attempt to read your code. 
To this end, comments are a quick explanation of the code. 
There are two kinds of comments, \Keyword{single-line comments} and \Keyword{multi-line comments}.

Single-line comments typically come after a line of code. 
For a single-line comment, simply type a double slash \Code{//} at the end of the line, and follow it with whatever notes you like, preferably to explain what that line of code does.
Alternatively, the comment can start on a line of its own.
Here are some examples:

\noindent\begin{minipage}{\linewidth}\begin{lstlisting}
int count; // This variable was declared 
           // to count something
count = count + 1; //Increments count by 1

//Variable declared, and initialized to pi
float length = 3.14159; 
\end{lstlisting}\end{minipage}

Multi-line comments, sometimes called \Keyword{block comments}, are used when you have a lot to say. 
They begin with a slash star (\Code{/*}) and are ended by a star slash (\Code{*/}). 
Here is an example:

\noindent\begin{minipage}{\linewidth}\begin{lstlisting}
/* 
 *This is a
 *multi-line 
 *comment
 */
 
/* This is also a comment */ 
\end{lstlisting}\end{minipage}

Block comments do not need a star at the beginning of every line (as in the preceding example), but many programmers write it anyways, because it makes it easier to see and understand that ``this is still a comment, don't write code here.'' 
Some development environments will automatically color-code certain pieces of code, so comments might be gray, for example, and the \Code{*} at the beginning of each line might be unnecessary in that case. 
However, someone else may use a different development environment that does not use colors, so the stars can still improve readability.

Keep in mind when commenting, the point is to be clear and concise. 
Try to explain what's happening as accurately as possible, but try to keep it short. 
As you learn C++, use comments to explain what you're doing and why. 
You have to assume that the person reading your code needs an explanation for each non-trivial line. 

\noindent\begin{minipage}{\linewidth}\begin{lstlisting}
//Commenting at the beginning of the file
//Allows you to give a summary of your program
#include <iostream>
using namespace std;

int main()
{
  // This cout statement outputs to the screen
  cout << "Hello world" << endl; 	
  cout << "What's the date?" << endl;

  // Comments should be used to explain things that may 
  // not be obvious to someone other than you
  cin >> date; // Takes the date from the user
 
  /* 
   * You can also use comments to remind yourself of 
   * changes you want to make, e.g. 
   * "debug code past this point" 
   */
  return 0;
}
\end{lstlisting}\end{minipage}

\LevelD{Review Questions}

\begin{enumerate}
\item Comment each line of this code: \nopagebreak[4]

\noindent\begin{minipage}{\linewidth}\begin{lstlisting}
#include <iostream>

using namespace std;

int main()
{
  int time;
  cout << "Enter time \n";
  cin >> time;
  int answer = (32 * time * time) / 2;
  cout << "The distance is ";
  cout << answer;
  cout << " seconds\n";
  return 0;
}
\end{lstlisting}\end{minipage}

\item Fix this code by removing or modifying comments so that it runs and compiles as it should.

\noindent\begin{minipage}{\linewidth}\begin{lstlisting}
/* #include <iostream> includes the iostream*

using namespace std;

int main()
{
  int time;				         // A place to store the time 
  cout << "Enter time \n"; // Ask to enter the time
  cin >> time;             // Takes user input
  int answer = (32 * time * time) / 2; // Calculates it
  cout << "The distance is ";          /* Outputs
  cout << answer;                         the distance
  cout << " seconds\n";                   in seconds */
  return 0;
}
\end{lstlisting}\end{minipage}

\item Explain the purpose of commenting. How does it help you? Why would someone else need to be able to understand your code?

\item Write and properly comment your own simple program.

\item Go back to the program you wrote from the previous question. Add further comments that explain what's happening and share the commented code with a classmate or friend. Ask them if they understand what's happening from just the comments.

\item Add comments to the following code. \nopagebreak[4]

Note: Save percentages in hockey are shown to three decimal places and not multiplied by 100: .900 instead of 90\%. \nopagebreak[4]

\noindent\begin{minipage}{\linewidth}\begin{lstlisting}
#include <iostream>
#include <cstdlib>

using namespace std;

int main()
{
  double shots, goals, saves, save_perc;
  char cont;

  do {
    cout.unsetf(ios::fixed);
    cout.unsetf(ios::showpoint);

    cout << "Enter the number of shots on goal:\t";
    cin >> shots;
    cout << "Enter the number of goals scored:\t";
    cin >> goals;
    cout << endl;

    saves = shots - goals;

    save_perc = (saves / shots);

    cout << "If there were " << shots << " shots and " 
      << goals << " goals\n";
    cout << "then the goalie's save percentage was ";

    cout.setf(ios::fixed);
    cout.setf(ios::showpoint);
    cout.precision(3);

    cout << save_perc << endl << endl;

    cout << "Run again? Y/N\t";
    cin >> cont;
    cont = toupper(cont);
    cout << endl;

  } while (cont == 'Y');

    return 0;
}
\end{lstlisting}\end{minipage}

\end{enumerate}

%\LevelD{Further Reading}
%
%\begin{itemize}
%\item ~
%\item ~
%\item ~
%\end{itemize}	

