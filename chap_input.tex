When a programmer wants a user to enter data, such as the price of an item, he or she will use the \Code{cin} object, pronounced ``see-in'', in conjunction with \Code{>>}, the \Keyword{redirect operator} in the program. 
Let us look at the following code:

\begin{lstlisting}
#include <iostream>
using namespace std;
int main()
{
	int x = 0;   				      
	cout << "Please enter a value for x: " << endl; 
	cin >> x;
	return 0;
}
\end{lstlisting}

When you compile and run this code, here's what the output will look like:

\noindent \Code{please enter a value for x: }

As a user you may want to check the value that was entered. 
To do this, simply add an additional \Code{cout} statement like this:

\begin{lstlisting}
#include <iostream>
using namespace std;
int main()
{
	int x = 0;   				      
	cout << "Please enter a value for x: " << endl; 
	cin >> x;
	cout << "The value of x is: " << x;
	return 0;
}
\end{lstlisting}

The output of this code is:

\noindent \texttt{please enter a value for x: }

Suppose the user enters a value of 1 for x.

\noindent \texttt{The value of x is: 1}

As you can see, the value displayed is the one entered. 
This can be a very useful technique in troubleshooting the values of variables throughout a program. 
Do not be afraid to insert additional \Code{cout} statements throughout a program to check the values of variables when debugging. 
This can help in the debugging process and speed up catching errors. 

If you want to have a user input more than one value, just repeat the code for each individual variable:

\begin{lstlisting}
#include <iostream>
using namespace std;
int main()
{
	int x = 0;
	int y = 0;
	
	cout << "Please enter a value for x: " << endl;
	cin >> x;
	cout << "Please enter a value for y: " << endl;
	cin >> y;
	cout << "The value of x is: " << x << endl;
	cout << "The value of y is: " << y << endl;
	return 0;
}
\end{lstlisting}

We can't always trust that the user will input the correct data into a variable. 
For example, if a user was prompted to input an age into a variable of type \Code{int} but typed the character \Code{z}, the program would not behave properly because the user entered the wrong data type. 
We can check for improper input like this by using the \Code{cin.fail()} function in a conditional statement.
Look at the following code:

\begin{lstlisting}
#include <iostream>
using namespace std;
int main()
{
	int x = 0;
	int y = 0;
	
	cout << "Please enter a value for x: " << endl;
	cin >> x;
	if (cin.fail())
	{
		cout << "That is not a valid data type!";
	}
}
\end{lstlisting}

This introduction to \Code{cin} statements is only the beginning. 
They will get slightly more complicated after we introduce strings, arrays, and overloaded operators.

\LevelD{Review Questions}

\LevelD{Homework Questions}

\LevelD{Review Answers}

\LevelD{Homework Answers}

\LevelD{Further Reading}

\begin{itemize}
\item \url{http://www.cplusplus.com/reference/iolibrary}
\item \url{http://www.cplusplus.com/doc/tutorial/basic_io}
\item ~
\end{itemize}	

